\subsection{Alternative agreement indices}
As an alternative to limits of agreement, \citet{lin2002} proposes the use of
the mean square deviation is assessing agreement. The mean square
deviation is defined as the expectation of the squared differences
of two readings . The MSD is usually used for the case of two
measurement methods $X$ and $Y$ , each making one measurement for
the same subject, and is given by
\[
MSDxy = E[(x - y)^2]  = (\mu_{x} - \mu_{y})^2 + (\sigma_{x} -
\sigma_{y})^2 + 2\sigma_{x}\sigma_{y}(1-\rho_{xy}).
\]


\citet{Barnhart} advises the use of a predetermined upper limit
for the MSD value, $MSD_{ul}$, to define satisfactory agreement.
However, a satisfactory upper limit may not be properly
determinable, thus creating a drawback to this methodology.


\citet{Barnhart} proposes both the use of the square root of the
MSD or the expected absolute difference (EAD) as an alternative agreement indices. Both of these indices can be interpreted intuitively, being denominated in the same units of measurements as the original
measurements. Also they can be compare to the maximum acceptable
absolute difference between two methods of measurement $d_{0}$.
\[
EAD = E(|x - y|) = \frac{\sum |x_{i}- y_{i}|}{n}
\]

The EAD can be used to supplement the inter-method bias in an
initial comparison study, as the EAD is informative as a measure
of dispersion, is easy to calculate and requires no distributional
assumptions.

\citet{Barnhart} remarks that a comparison of EAD and MSD , using
simulation studies, would be interesting, while further adding
that `It will be of interest to investigate the benefits of these
possible new unscaled agreement indices'. For the Grubbs' `F vs C' and `F vs T' comparisons, the inter-method bias, difference variances, limits of agreement and EADs are shown
in Table 1.5. The corresponding Bland-Altman plots for `F vs C' and `F vs T' comparisons were depicted previously on Figure 1.3. While the inter-method bias for the `F vs T' comparison is smaller, the EAD penalizes the comparison for having a greater variance of differences. Hence the EAD values for both comparisons are much closer.
\begin{table}[ht]
\begin{center}
\begin{tabular}{|c|c|c|}
  \hline
 & F vs C & F vs T  \\
  \hline
Inter-method bias & -0.61 & 0.12 3 \\
Difference variances & 0.06 & 0.22  \\
Limits of agreement & (-1.08,	-0.13) & (-0.81,1.04) \\
  EAD & 0.61 & 0.35  \\
   \hline
\end{tabular}
\caption{Agreement indices for Grubbs' data comparisons.}
\end{center}
\end{table}

Further to  \citet{lin2000} and \citet{lin2002}, individual agreement between two measurement methods may be
assessed using the the coverage probability (CP) criteria or the total deviation index (TDI). If $d_{0}$ is predetermined as the maximum acceptable absolute difference between two methods of measurement, the probability that the absolute difference of two measures being less than $d_{0}$ can be computed. This is known as the coverage probability (CP).

\begin{equation}
CP = P(|x_{i} - y_{i}| \leq d_{0})
\end{equation}

If $\pi_{0}$ is set as the predetermined coverage probability, the
boundary under which the proportion of absolute differences is
$\pi_{0}$ may be determined. This boundary is known as the `total
deviation index' (TDI). Hence the TDI is the $100\pi_{0}$
percentile of the absolute difference of paired observations.

\end{document}
